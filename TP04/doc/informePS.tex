% !TEX encoding = UTF-8 Unicode
\documentclass{sig-alternate}
\usepackage{textcomp}
\usepackage{graphics}
\usepackage[utf8]{inputenc}
\usepackage[T1]{fontenc}

\hyphenation{o-pe-ra-do-res}

\begin{document}

\pagenumbering{arabic}

\title{Algoritmos Genéticos}
\subtitle{Sistemas de Inteligencia Artificial - ITBA}

\numberofauthors{3}

\author{
	\alignauthor{Carlos Sessa}\\
	\alignauthor{Lucas Pizzagalli}\\
	\alignauthor{Nicolás Purita}\\	
}

\date{12 de Junio de 2012}

\maketitle

\section{Introducción}

	Se implementó un motor de algoritmo genéticos para obtener los pesos para la red neuronal construida en el Trabajo Práctico número 2.
	La red neuronal resuelve la función que se puede observar en la figura \ref{fig:function}:

	\begin{figure}[!ht]
		\includegraphics[scale=0.5]{./figures/function.png}
  		\caption{Distribución de puntos dada}
  		\label{fig:function}
	\end{figure}

	En la figura \ref{fig:function} se puede observar que los puntos de la entrada pertenecen al intervalo $[-3.5, 3.5]$ y la salida se encuentra en el intervalo $(0, 1)$.\\
	El algoritmo genético se implementó en \textit{Java} y la red neuronal fue realizada en \textit{Matlab}. Utilizando el \textit{MATLAB Compiler Runtime} se realizan las pruebas pertinentes para verificar el funcionamiento de la red.\\

\section{Desarrollo y Problemas encontrados}

	\subsection{Distintas arquitecturas}

	Mediante el archivo de configuración se puede seleccionar la arquitectura que se desea utilizar para realizar las pruebas. Las arquitecturas a elegir son las siguientes:

	\begin{center}
		\begin{enumerate}
			\item $[30\,\,20\,\,10\,\,1]$
			\item $[10\,\,10\,\,1]$
			\item $[10\,\,\,1]$
			\item $[10\,\,10\,\,10\,\,10\,\,1]$
			\item $[40\,\,20\,\,1]$
			\item $[10\,\,10\,\,10\,\,1]$
			\item $[5\,\,10\,\,20\,\,1]$
		\end{enumerate}
	\end{center}

	\subsection{Representación del individuo}

	La representación más óptima del individuo fue representar a toda la red como un vector de \textit{doubles}. Esta representación no se adapta a la representación de la matriz en \textit{MATLAB}, por lo tanto se realiza una transformación del individuo. Esta transformación consiste en cambiar el vector de doubles a una matriz y viceversa.

	\subsubsection{Fidelidad del individuo}

	\begin{itemize}

		\item \textbf{Completitud}: Es completa ya que se puede representar todo el dominio del problema.

		\item \textbf{Coherencia}: Representa únicamente el dominio del problema, ya que ninguna representación puede pertenecer a un conjunto fuera del dominio.

		\item \textbf{Uniformidad}: Se puede concluir que esta representación es uniforme ya que es imposible representar dos individuos distintos con la misma cadena, en caso que esto sucediera esa representación sería exactamente la misma que la otra.	

		\item \textbf{Sencillez}: Convertir matriz de pesos a un vector de doubles es una operación simple.

		\item \textbf{Localidad}: Es local dada que un cambio en un elemento
		del vector genera un cambio en el peso de la conexión que representa.

	\end{itemize}
	
\section{Resultados y Conclusiones}
	Los resultados se van a presentar en dos secciones distintas.
	La primera no utilizará el operador backpropagation y la segunda sí. \\

	Con el fin de obtener resultados comparables entre sí se define un
	contexto base para todas las pruebas de cada sección.
	Luego en cada prueba se reemplaza el caso base con lo que se desea.
	Debe tenerse en cuenta que el hecho de prefijar algunos parámetros puede
	hacer que los distintos métodos y operadores a comparar se vean
	beneficiados o perjudicados por su funcionamiento en conjunto. \\

	Cabe destacar que aparte de los operadores pedidos por la cátedra se
	implementó un operador de cruce adicional al que denominamos \textit{Gene}.
	\textit{Gene} funciona de manera similar al cruce clásico pero en vez de
	realizar el cruce en base a un \textit{locus} lo hace reemplazando
	capas enteras.


	\subsection{Sin backpropagation}
	Primero se intenta resolver el problema sin usar el operador backpropagation.
	El contexto base usado en esta sección es el siguiente:

	\begin{itemize}
		\item Tamaño de la población es de 52 individuos.
		\item La brecha generacional (\textit{G}) es de 0.5.
		\item La arquitectura elegida para realizar las pruebas es la de [10 10 1].
		\item El operador de mutación inicial es \textit{Clásico} con una probabilidad de 0.5 de mutar.
		\item El operador de cruce inicial es \textit{Clásico} con una probabilidad de 1 de cruce.
		\item El operador \textit{Backpropagation} se encuentra desactivado es decir que se ejecute con probabilidad 0.
		\item El método de selección es \textit{Elite}.
		\item El método de reemplazo es \textit{Elite} al igual que el de selección.
		\item Hay dos condiciones de corte, ya sea por \textit{Máxima cantidad de generaciones} (500) y/o \textit{Contenido} (donde la última verifica que si en 50 épocas no mejoro mas de un 0.01 el fitness del individuo finaliza su ejecución).
	\end{itemize}

	\subsubsection{Mutación clásica}
	Primero se prueba variar la probabilidad de la mutación clásica.
	Los resultados se pueden observar en la tabla \ref{table:simple_mutation_prob}.

	\subsubsection{Mutación no uniforme}
	Luego se prueba variar la probabilidad de la mutación no uniforme.
	Los resultados se pueden observar en la tabla \ref{table:simple_mutation_no_uniform}.

	\subsubsection{Cruce}
	Por último se corre el algoritmo usando distintos cruces.
	Los resultados se encuentran en la tabla \ref{table:crossover}.

	\subsubsection{Análisis de resultados}
	Después de probar con distintas configuraciones, notamos que sin el
	operador de backpropagation no conseguimos buenos	resultados.
	Entendemos que esto sucede porque el individuo es muy largo haciendo
	que el algoritmo tarde mucho en encontrar un resultado útil.

	\subsection{Con backpropagation}
	En esta sección se prueban distintas configuraciones utilizando el
	operador backpropagation.

	\subsection{Elección del operador de Cruza}

	\subsection{Elección del operador de Mutación}

	\subsection{Utilización de backpropagation}

	\subsection{Elección del operador de Selección y reemplazo}

\section{Posibles mejoras}

	Como posible mejora se concluye que el operador \textit{backpropagation}
	podría disminuir su probabilidad de acción a medida que se alcanza un
	fitness deseado.
	El motivo por el cual se desea disminuir esa probabilidad es para que
	comience a tener más influencia los operadores de mutación y cruce.
	El deseo de que la mutación y cruce tenga más influencia a lo largo de
	las generaciones es porque se observó que el fitness en un punto se
	asintotiza al error cuadrático medio obtenido por \textit{backpropagation}
	en el Trabajo Práctico Especial 2.

\onecolumn

\begin{table}[htp]
	\begin{center}
	\begin{tabular}{|c|c|c|c|}
		\hline
	     Fitness & Generación & Corte & $p_{m}$ \\
		\hline
		0.22501570790437642 & 60 & Contenido & 0.05 \\
		0.21842640558999668 & 60 & Contenido & 0.1 \\
		0.22113522454399956 & 67 & Contenido & 0.3 \\ 
		0.22050756611447994 & 72 & Contenido & 0.5 \\
		\hline
	\end{tabular}
	\caption{Configuración simple variando la $p_m$}
	\label{table:simple_mutation_prob}
	\end{center}
\end{table}

\begin{table}[htp]
	\begin{center}
	\begin{tabular}{|c|c|c|c|c|p{1cm}|}
		\hline
	     Fitness & Generación & Corte & $p_{m}$ & decrecimiento (\%) & dec. gen. \\
		\hline
		0.21222349890474348 & 50 & Content & 0.6 & 0.05 & 30 \\
		0.22795744299992385 & 77 & Content & 0.6 & 0.1  & 30 \\
		0.21740760693242447 & 65 & Content & 0.6 & 0.15 & 30 \\
		\hline
	\end{tabular}
	\caption{Configuración simple método de mutación No Uniforme}
	\label{table:simple_mutation_no_uniform}
	\end{center}
\end{table}

\begin{table}[htp]
	\begin{center}
	\begin{tabular}{|c|c|c|c|c|}
		\hline
	     Fitness & Generación & Corte & $p_{m}$ & Cruce \\
		\hline
		0.22501570790437642 & 60 & Contenido & 0.05 & Clásico \\
		0.20662153685787393 & 50 & Contenido & 0.05 & Gene \\
		0.21418198155450324 & 50 & Contenido & 0.05 & Multiples puntos con 2 \\
		0.21674758534799746 & 70 & Contenido & 0.05 & Uniforme con 0.3 prob \\
		0.20662153685787393 & 50 & Contenido & 0.05 & Anular \\
		\hline
	\end{tabular}
	\caption{Configuración simple variando el método de cruce}
	\label{table:crossover}
	\end{center}
\end{table}


\end{document}
