% !TEX encoding = UTF-8 Unicode
\documentclass{sig-alternate}
\usepackage{textcomp}
\usepackage{graphics}
\usepackage[utf8]{inputenc}
\usepackage[T1]{fontenc}
\usepackage{listings}
\usepackage{subfigure}

\hyphenation{o-pe-ra-do-res con-fi-gu-ra-ción po-bla-ción com-por-ta-mien-to in-di-vi-duos res-pec-ti-va-men-te}

\begin{document}

\pagenumbering{arabic}

\title{Algoritmos Genéticos}
\subtitle{Sistemas de Inteligencia Artificial - ITBA}

\numberofauthors{3}

\author{
	\alignauthor{Carlos Sessa}\\
	\alignauthor{Lucas Pizzagalli}\\
	\alignauthor{Nicolás Purita}\\	
}

\date{12 de Junio de 2012}

\maketitle

% \section{Introducción}

% 	Se implementa un motor de algoritmos genéticos para obtener los pesos para la red neuronal construida en el Trabajo Práctico Especial Número 2.
% 	La red neuronal resuelve la función que se puede observar en la figura \ref{fig:function}.

% 	\begin{figure}[!ht]
% 		\includegraphics[scale=0.5]{./figures/function.png}
%   		\caption{Distribución de puntos dada}
%   		\label{fig:function}
% 	\end{figure}

% 	En la figura \ref{fig:function} se puede observar que los puntos de la entrada pertenecen al intervalo $[-3.5, 3.5]$ y la salida se encuentra en el intervalo $(0, 1)$.\\
% 	El algoritmo genético se implementó en \textit{Java} y la red neuronal fue realizada en \textit{Matlab}. Utilizando el \textit{MATLAB Compiler Runtime} se realizan las pruebas pertinentes para verificar el funcionamiento de la red.\\

% \section{Desarrollo y Problemas encontrados}

% Se han implementado varios operadores genéticos, métodos de selección, condiciones de corte y demás configuraciones que se presentarán en las secciones a continuación.
% Con el objetivo de facilitar la configuración se ha creado un archivo de \textit{properties} en el que se debe detallar como se desea entrenar la red.
% Un ejemplo de dicho archivo se puede observar en la lista \ref{code:simple}.

% 	\subsection{Operadores genéticos}

% 	\subsubsection{Mutación}
% 	Las mutaciones se realizan tomando un número aleatorio definido entre $-1.5$ y $1.5$.
% 	Se han implementado las mutaciones del tipo clásicas y no uniformes.

% 	\subsubsection{Crossover}
% 	Se han decidido implementar los siguientes algoritmos de cruzamiento:
% 	\begin{itemize}
% 		\item cruce de un punto (clásica)
% 		\item cruce de dos puntos
% 		\item cruce anular
% 		\item cruce uniforme parametrizado
% 		\item Gene
% 	\end{itemize}

% 	\subsubsection{Backpropagation}
% 	Otro operador utilizado es \textit{Backpropagation}.

% 	El mismo toma como parámetro la cantidad de épocas que debe entrenar la red y la probabilidad con la que se debe aplicar.

% 	\subsection{Métodos de selección y reemplazo}

% 	Para ambos casos se han desarrollado los métodos descriptos a continuación, los cuales pueden ser combinados:
	
% 	\begin{itemize}
% 		\item Método de Elitismo
% 		\item Método de Ruleta
% 		\item Método de Torneos
% 		\item Método de Boltzman
% 		\item Método de selección universal estocástica
% 	\end{itemize}

% 	\subsection{Distintas arquitecturas}

% 	Mediante el archivo de configuración se puede seleccionar la arquitectura que se desea utilizar para realizar las pruebas. Las arquitecturas a elegir son las siguientes:

% 	\begin{center}
% 		\begin{enumerate}
% 			\item $[30\,\,20\,\,10\,\,1]$
% 			\item $[10\,\,10\,\,1]$
% 			\item $[10\,\,\,1]$
% 			\item $[10\,\,10\,\,10\,\,10\,\,1]$
% 			\item $[40\,\,20\,\,1]$
% 			\item $[10\,\,10\,\,10\,\,1]$
% 			\item $[5\,\,10\,\,20\,\,1]$
% 		\end{enumerate}
% 	\end{center}

% 	\subsection{Representación del individuo}

% 	Con el objetivo de representar al individuo, se ha elegido transformar
% 	la matriz de pesos a un vector de números \textit{doubles} como se muestra en
% 	la figura \ref{fig:indiv} y de esta forma poder aplicar los operadores
% 	pertinentes de una forma rápida y sencilla. \\

% 	Esta representación no se adapta a la representación de la matriz en
% 	\textit{MATLAB}, por lo tanto se realiza una transformación del individuo.
% 	Esta transformación consiste en cambiar el vector de doubles por una
% 	matriz y viceversa. \\

% 	\subsubsection{Fidelidad del individuo}

% 	\begin{itemize}

% 		\item \textbf{Completitud}: Es completa ya que se puede representar todo el dominio del problema.

% 		\item \textbf{Coherencia}: Representa únicamente el dominio del problema, ya que ninguna representación puede pertenecer a un conjunto fuera del dominio.

% 		\item \textbf{Uniformidad}: Se puede concluir que esta representación es uniforme ya que es imposible representar dos individuos distintos con la misma cadena, en caso que esto sucediera esa representación sería exactamente la misma que la otra.	

% 		\item \textbf{Sencillez}: Convertir matriz de pesos a un vector de doubles es una operación simple.

% 		\item \textbf{Localidad}: Es local dada que un cambio en un elemento
% 		del vector genera un cambio en el peso de la conexión que representa.

% 	\end{itemize}

	\section{Consideraciones generales}

		\subsection{Cálculo del Fitness}

			El fitness del individuo es obtenido mediante una simple división del \textit{Error cuadrático medio} que se obtiene de la red al evaluar los puntos.
			La fórmula para el mismo es:

			\begin{equation}
				\textbf{Fitness} = \dfrac{1}{ECM}
			\end{equation}

			donde \textit{ECM} es el Error Cuadrático Medio.

		\subsection{Selección de valores para Boltzman}

			Para elegir los valores iniciales de las temperaturas que se utilizan en el algoritmo de Boltzman (Temperatura mínima, máxima y decremento por generación) se eligieron ciertos valores que cumplan con el siguiente criterio. Al inicio de la evolución todos los individuos deben tener una probabilidad similar de ser seleccionados entre toda la población y a medida que avanzan las generaciones los individuos con mayor \textit{fitness} deben tener mayor probabilidad de ser seleccionados.
			Los valores que se eligieron de Boltzman para hacer las pruebas son variables dependiendo de la cantidad de iteraciones que se desea realizar, y suponiendo un \textit{fitness} en un rango determinado. Esto logra una gran diversidad en un comienzo, y luego va refinando la población descartando de a poco los individuos menos aptos. 	

	\section{Resultados}

		Con el fin de obtener resultados comparables entre sí se define un
		contexto base para todas las pruebas, con la intención de obtener distintos 
		resultados se realizan cambios sobre esta configuración:

		\begin{itemize}
			\item \textbf{Cantidad de individuos:} 52.
			\item \textbf{Mutación:} Clásica. Con una probabilidad de mutar un individuo de $0.5$ 
			y de mutar un locus de $0.03$.
			\item \textbf{Cruce:} Clásico.
			\item \textbf{Backpropagation:} 0 (Desactivado)
			\item \textbf{Selección:} Elite+Rulette (Seleccionando 5 para Elite y 23 para Rulette).
			\item \textbf{Reemplazo:} Elite+Rulette (Seleccionando 6 para Elite y 18 para Rulette).
			\item \textbf{Criterios de corte:} Máxima 1000 generaciones o Contenido, si en 60 generaciones
			no mejora un $1\%$ el mejor individuo corta.
		\end{itemize}

		\subsection{Boltzman}

		Con el objetivo de analizar el desempeño de este método de selección y reemplazo, se decidió probar con una configuración Boltzman-Boltzman (donde la selección y el reemplazo se realiza con dicho método) con una configuración como la siguiente:
			
		\begin{enumerate}			
			\item \textit{Temperatura inicial}: 40
			\item \textit{Temperatura mínima}: 0.4
			\item \textit{Decremento por generación}: 0.05
			\label{eq:boltz}
		\end{enumerate}

		Esto quiere decir que la temperatura mínima se alcanza en 800 generaciones aproximadamente, 	o sea que entre la primera y la generación número 800, se van seleccionando cada vez más proporción de individuos con \textit{fitness} más alto.\\
		Como se puede observar en la figura \ref{fig:boltzboltz}, tanto el valor medio como el mejor \textit{fitness} tienden a empeorar, comportarse de una forma bastante errática y luego comenzar a mejorar de a poco. Esto se debe a que, como se ha descripto anteriormente, en un comienzo se tiende a elegir de una manera casi totalmente aleatoria, pero a medida que se acerca y supera la generación 600 y aun más claramente cuando se alcanza la temperatura mínima, la selección se realiza más a conciencia, dándole prioridad a los individuos con más aptitud. 
		
		\subsection{Mutación No uniforme}
		
		Con el objetivo de comparar el comportamiento del sistema según el decaimiento de la probabilidad de mutación a medida que las generaciones avanzan. Por lo tanto, se decidió comparar mutaciones no uniformes con decaimientos de $5\%$, $10\%$ y $15\%$ cada 30 generaciones. Los resultados de las mismas se pueden observar en las figuras \ref{fig:mut_no_un_5} , \ref{fig:mut_no_un_10} y \ref{fig:mut_no_un_15} respectivamente. De las mismas se puede observar a simple vista, como en todos los casos termina por contexto (60 generaciones con un cambio menor a $1\%$), sin embargo, se puede ver como al ser el decremento mayor, el algoritmo termina antes. Esto se debe a que al haber muchas menos mutaciones, no hay introducción de elementos nuevos y no se llega a cambiar lo suficiente las redes como para obtener nuevas combinaciones y se consigan mejores resultados. Este estancamiento, también se evidencia en el menor desvío estándar.

		\subsection{Crossover}

		Se realizaron varias corridas distintas en donde se modifica unicamente el método de \textit{cruce}. En la figura \ref{fig:crossover_comparison} se pueden observar los cuatro métodos distintos. Se observa en esta comparación que el mejor resultado obtenido es con el método de cruce \textit{Uniforme}.\\
		El método de cruce aporta diversidad a la población para poder explorar mayor espacio de individuos.

		\subsection{Selección y Reemplazo}

		En esta sección se obtuvieron varios resultados donde se pueden sacar algunas conclusiones acerca de los métodos de selección y reemplazo.\\
		En primer lugar el comportamiento del algoritmo utilizando como método de Selección y Reemplazo \textbf{Ruleta}, se puede observar como el \textit{fitness} oscila. Este comportamiento es de esperar ya que este método no garantiza el mejor individuo para la próxima generación sino que tiene más probabilidad de ser seleccionado. En la figura \ref{fig:rulette_rulette} se observa este comportamiento explicado.\\
		Otro punto a destacar es que si utilizamos el método \textit{Elite} (como selección y reemplazo) no aporta diversidad a la población, dado que siempre selecciona los \textbf{N} mejores individuos para la siguiente generación, provocando una prematura perdida de la diversidad y un amesetamiento de los resultados como se puede comprobar en la figura \ref{fig:elite_elite}. Por lo tanto la utilización de \textit{Elite} en conjunto con \textit{Boltzman} o \textit{Rulette} logran aportar diversidad a la población y de esta forma obtener mejores resultados como se pueden observar en las figuras \ref{fig:elite_boltzman2} y \ref{fig:elite_rulette2} respectivamente. Ésta última configuración genera los mejores resultados obtenidos sin la utilización del operador \textit{Backpropagation} obteniendo un \textit{fintess} de aproximadamente $7$.

		\subsection{Backpropagation vs Algoritmos Genéticos con Backpropagation}

		En las figuras \ref{fig:onlyBP} y \ref{fig:AGWithBP} se puede observar como mejora el comportamiento de la \textit{Red neuronal} utilizando solo \textit{Backpropagation} y combinandolo con algoritmos genéticos  respectivamente.
		El primer punto a destacar es el comportamiento de la curva del \textit{Fitness}.
		Se puede observar que ambas configuraciones brindan un buen desempeño. Por un lado, usando únicamente \textit{Backpropagation} la curva posee más oscilaciones. Esto se debe a que por la naturaleza del algoritmo, no se guarda la red de mayor desempeño entre épocas, habiendo probabilidad de perder la misma como se puede observar en las caídas del \textit{fitness} en la figura  \ref{fig:onlyBP}.\\
		Por otro lado, al combinar dicho método con algoritmos genéticos y los correctos métodos de selección y reemplazo, podemos mantener siempre al mejor individuo, generando una curva del mejor desempeño monótona creciente y a su vez, gracias a los cruces y mutaciones, obtener un \textit{fitness} mayor como se puede comprobar en la figura   \ref{fig:AGWithBP}.
		
		
		\section{Conclusiones}
		
		Se puede concluir que el algoritmo genético puede resolver el problema sin la utilización del operador \textit{Backpropagation}, sin embargo si se utiliza dicho operador, los resultados obtenidos son mejores.
		A su vez, es importante destacar que la combinación entre algoritmos genéticos y \textit{Backpropagation} tiene un mejor desempeño que al utilizar este último solo.

\clearpage
\onecolumn


\begin{figure}[!ht]
	\begin{center}
			\framebox{\includegraphics[scale=0.7]{./figures/boltzmanboltzman.png}}
	\end{center}
	\caption{Selección y reemplazo por método de Boltzman}
	\label{fig:boltzboltz}
\end{figure}

\clearpage
\begin{figure}[!ht]
	\begin{center}
		\subfigure[Decaimiento de 5\% cada 30 generaciones ]
			{\label{fig:mut_no_un_5}\framebox{\includegraphics[scale=0.44]{./figures/not_uniform_5.png}}}
		\hspace{20pt}
		\subfigure[Decaimiento de 10\% cada 30 generaciones ]
			{\label{fig:mut_no_un_10}\framebox{\includegraphics[scale=0.45]{./figures/not_uniform_10.png}}}
		\hspace{20pt}
		\subfigure[Decaimiento de 15\% cada 30 generaciones ]
			{\label{fig:mut_no_un_15}\framebox{\includegraphics[scale=0.45]{./figures/not_uniform_15.png}}}
		\hspace{20pt}
	\end{center}
	\caption{Influencia del decaimiento en mutaciones no uniformes }
	\label{fig:not_uniform}
	
\end{figure}

\clearpage
\begin{figure}[!ht]
	\begin{center}
		\subfigure[Crossover Classic]
			{\label{fig:crossover_classic}\framebox{\includegraphics[scale=0.35]{./figures/crossover_classic.png}}}

		\subfigure[Crossover Multiple Point con dos puntos]
			{\label{fig:crossover_multipoint}\framebox{\includegraphics[scale=0.35]{./figures/crossover_multipoint.png}}}

		\subfigure[Crossover Uniform]
			{\label{fig:crossover_multipoint}\framebox{\includegraphics[scale=0.30]{./figures/crossover_uniform.png}}}

		\subfigure[Crossover Anular]
			{\label{fig:crossover_anular}\framebox{\includegraphics[scale=0.35]{./figures/crossover_anular.png}}}

	\end{center}
	\caption{Comparación entre distintos métodos de crossover}
	\label{fig:crossover_comparison}
\end{figure}

\clearpage
\begin{figure}[!ht]
	\begin{center}
			\framebox{\includegraphics[scale=0.7]{./figures/rulette_rulette.png}}
	\end{center}
	\caption{Selección y reemplazo por método de Ruleta}
	\label{fig:rulette_rulette}
\end{figure}

\begin{figure}[!ht]
	\begin{center}
		\subfigure[Elite-Elite]
			{\label{fig:elite_elite}\framebox{\includegraphics[scale=0.5]{./figures/elite_elite.png}}}
		\hspace{20pt}
		\subfigure[Elite/Ruleta - Elite/Ruleta]
			{\label{fig:elite_rulette2}\framebox{\includegraphics[scale=0.5]{./figures/elite_rulette2.png}}}
		\subfigure[Elite/Boltzman - Elite/Boltzman]
			{\label{fig:elite_boltzman2}\framebox{\includegraphics[scale=0.5]{./figures/elite_boltzman2.png}}}	
	\end{center}
	\caption{Comparación Elite, Elite/Ruleta y Elite/Boltzman}
	\label{fig:comparisonElite}
\end{figure}


\begin{figure}[!ht]
	\begin{center}
		\subfigure[Únicamente Backpropagation]
			{\label{fig:onlyBP}\framebox{\includegraphics[scale=0.4]{./figures/OnlyBPFitness.png}}}
		\hspace{20pt}
		\subfigure[Algoritmo Genético con Backpropagation]
			{\label{fig:AGWithBP}\framebox{\includegraphics[scale=0.7]{./figures/BPFitness.png}}}
	\end{center}
	\caption{Comparación entre Backpropagation y Algoritmo Genéticos con Backpropagation}
	\label{fig:comparinBP}
\end{figure}

\clearpage

\begin{lstlisting}[caption={Archivo de configuración simple},label={code:simple}]

popSize = 52
architecture = 2
generationGap = 0.5

mutation = Classic
mutationProbability = 0.5
Mutation.alleleProb = 0.03

Backpropagation.probability = 0

crossover = Classic
crossoverProbability = 1


selection = Elite / Rulette
Elite.toSelect = 5
Rulette.toSelect = 23

replacement = Elite / Rulette
replacement.Elite.toSelect = 6
replacement.Rulette.toSelect = 18

ending = MaxGeneration / Content
MaxGeneration.iterationToEnd = 500

Content.improvement = 0.01
Content.iterationToImprove = 60
Content.iterationToImprove = 15

\end{lstlisting}


\end{document}
