\documentclass{sig-alternate}
\usepackage{textcomp}
\usepackage{graphics}
\pagestyle{plain}
\usepackage{subfigure}
\usepackage[margin=10pt,font=small,labelfont=bf, labelsep=endash, skip=0pt]{caption}
\usepackage[latin1]{inputenc}
\usepackage{listings}

\begin{document}

\pagenumbering{arabic}

\title{Mahjong Solitaire}
\subtitle{Sistemas de Inteligencia Artifical - ITBA}

\numberofauthors{3}

\author{
	\alignauthor{Carlos Sessa}\\
	\alignauthor{Lucas Pizzagalli}\\
	\alignauthor{Nicol\'as Purita}\\
}

\date{09 de Mayo de 2011}

\maketitle

\section*{Introducci\'on}
	Se implementa un \textit{Sistema de Producci\'on} el cual es utilizado para resolver un juego denominado \textbf{Mahjong-Solitaire}, tambien conocido como el \textit{Taipei}. Se utiliza un motor de inferencia en \textit{Java} para la resoluci\'on del problema designado, el cual fue entregado por la c\'atedra. \\
	Como punto de partida para crear el \textit{Sistema de Producci\'on} se dise\~{n}a un sistema de almacenamiento apropiado que represente el problema de la forma m\'as \'optima. Se utilizan 3 estrat\'egias de b\'usqueda no informadas para comparar las distintas soluciones alcanzadas, las cuales se detallan en el transcurso del informe

\section*{Reglas del Mahjong-Solitaire}
	El juego \textbf{Mahjong-Solitaire} consiste en eliminar todas las fichas de un tablero de 144 fichas. \\
	Una ficha puede ser removida siempre que no se encuentre bloqueada y ademas cumpla con la condici\'on de que las fichas son del mismo tipo. Las fichas existentes son:

\begin{table}[h]
\begin{center}
	\begin{tabular}{|c|c|c|}
	\hline
	 Tipo de Ficha & Identificador & Cantidad\\
	\hline \hline
	\textit{Character} & 9 & 4 de c$\setminus$u \\
	\textit{Bamboo} & 9 & 4 de  c$\setminus$u  \\
	\textit{Circles} & 9 & 4 de  c$\setminus$u  \\
	\textit{Dragons} & 3 & 4 de  c$\setminus$u  \\	
	\textit{Winds} & 4 & 4 de  c$\setminus$u  \\
	\textit{Season} & 4 & 1 de c$\setminus$u \\
	\textit{Flowers} & 4 & 1 de  c$\setminus$u  \\
	\hline
	\end{tabular}
\end{center}
\caption{Distribuci\'on de fichas}
\label{tab:tiles}
\end{table}

	Existe una excepci\'on en la remoci\'on de las fichas del tipo \textit{Season} o \textit{Flowers}. La excepci\'on esta basada en que cualquier par de fichas \textit{Season} o \textit{Flowers} puede ser removidas utilizando cualquier combinaci\'on entre las mismas, esto se debe a que existen unicamente 4 fichas de cada tipo. \\
	Una ficha bloqueada puede estar en forma \textit{Horizontal}, \textit{Vertical} o \textit{Ambos}. Estos bloqueos se definen del siguiente modo:
	\begin{itemize}
		\item \textbf{Horizontal}: La ficha elegida posee una ficha a su izquierda y derecha.
		\item \textbf{Vertical}: La ficha elegida posee una ficha encima de ella.		
	\end{itemize}
	En la Figura \ref{fig:blocking} se puede observar distintas posibilidades de bloqueos.
\section*{Desarrollo}

	Se representa un estado del juego como un vector de vectores en donde cada celda posee una \textit{Tile}. Esta vector de vectores es del tipo:
	\begin{itemize}
		\item \textit{Tile}[a][b][c]
	\end{itemize}
	donde \textit{a} indica el nivel en el que se encuentra la ficha, \textit{b} es la fila donde se encuentra y \textit{c} la columna. La clase \textit{Tile} esta definida como la composici\'on de un tipo de ficha y un n\'umero.
	
\section*{Funciones de costo}

	Dado que el objetivo del juego es simplemente retirar todas las fichas del tablero, la funci\'on de costo propuesta es la cantidad de fichas sacadas en cada estado. \\
	Esta funci\'on de costo es variable si buscamos la soluci\'on utilizando un ordenador, dado que se pueden aplicar varias reglas en simult\'aneo seg\'un un criterio de selecci\'on de pares de fichas. \\
	Caso contrario de esta funci\'on de costo en la vida real ya que en cada estado del juego se pueden remover unicamente 2 fichas en simult\'aneo, por lo tanto esta funci\'on se comporta en forma constante.

\section*{Heur\'isticas}

	Se considera \textit{Pair} a todo par de fichas posibles a retirar del tablero en un estado y \textit{Payers} a todo conjunto de \textit{Pares} de fichas posibles a retirar en un estado del juego. \\
	Se proponen las siguientes heur\'isticas (para desarrollar en las estrat\'egias de b\'usquedas informadas):
	\begin{itemize}

		\item \textbf{M\'as Alta} \textit{(h1)}: Se libera el \textit{Pair} de m\'as alto nivel. En el mejor de los casos la remoci\'on de una ficha en el nivel m\'as alto puede liberar como m\'aximo 4 fichas, los cual explota nuevas posibilidades de generar mas \textit{Payers}.
		
		\item \textbf{Mayor cantidad de fichas horizontales} \textit{(h2)}: Esta heur\'istica busca reducir la posibilidad de eliminar una fila del tabler. La heur\'istica se basa en calcular la cantidad de fichas que posee cada una del \textit{Pair} en la fila en la que se encuentra, esto se calcula como:
			\begin{eqnarray}
				X_{1}  & = & \# Fichas_{izq}+ \# Fichas_{der} \\
				X_{2}  & = & \# Fichas_{izq}+ \# Fichas_{der} \\
				\textbf{h2} & = & X_{1} + X{2}
			\end{eqnarray}
		donde $X_{1}$ y $X_{2}$ es un \textit{Pair}. Esta heur\'istica se basa en reducir la posibilidad de eliminar las filas que se encuentran en los niveles mas bajos, debido a que no liberan nuevos \textit{Payers}.
		
		\item \textbf{Generador M\'aximo de \textit{Payers}} \textit{(h3)}: Esta heur\'isitca se basa en eliminar los \textit{Pairs} que generen mayor cantidad de \textit{Payers} en el pr\'oximo estado.
		
	\end{itemize}

\section*{Resultados y Conclusiones}	
En la tabla \ref{tab:cost} se ven los distintos resultados para distintas estrat\'egias de b\'usquedas para algoritmos no informados. \\
Se ve que para la estrategia \emph{DFS} se logran mejores resultados que para \emph{BFS} y \emph{Profundizaci\'on iterativa}, en el sentido del tiempo promedio y nodos expandidos, dado que todos llegan a la soluci\'on en un tablero de nivel f\'acil ya que en tableros mas complejos (con 144 fichas) no se alcanza una soluci\'on. Esto podr\'ia deberse a que para el problema que se trata, siempre se necesita ir hasta el final del \'arbol de b\'usqueda, dado que se busca el camino que cubra todo el tablero y no m\'as de all\'i. Es por eso que si \emph{DFS} entra en la rama de la soluc\'ion, seguro llega a esta y no expande m\'as nodos. En cambio, \emph{BFS} no encontrar\'a la soluci\'on hasta explorar el \'ultimo nivel y lo mismo sucede con \emph{Profundizaci\'on iterativa}. Por estos motivos, las estrategias de b\'usqueda informadas son comparadas con \emph{DFS}. \\
En los tableros \ref{fig:layout3} y \ref{fig:layout4} utilizando \emph{BFS} y \emph{Profundizaci\'on Iteratica} se observa que es necesario expandir todo el \'arbol, por lo que en un tiempo aproximado de 5 minutos no se puede llegar a la soluci\'on. De todos modos se realizaron pruebas con ord\'enes de magnitud mas grandes a este tiempo (7 horas y 30 minutos aprox.) y se concluy\'o a que \emph{BFS} no termina ya que debe expandir hasta el \'ultimo nivel para obtener una soluci\'on.

\onecolumn

\begin{figure}[h!]
  \begin{center}
  	\includegraphics[scale=0.3]{images/blocking.png}
  \end{center}
  \caption{Ficha 2 es bloqueada por la 1 y 3. La ficha 4-7 estan bloqueadas por la 8. El resto de las fichas son movibles.}
  \label{fig:blocking}
\end{figure}

\begin{figure}[h!]
  \begin{center}
      \subfigure[Tablero 1]{\label{fig:layout1}\includegraphics[scale=0.3]{Boards/Layout1.png}}
    \hspace{20pt}
    \subfigure[Tablero 2]{\label{fig:layout3}\includegraphics[scale=0.2]{Boards/Layout3.png}}
    \hspace{20pt}
    \subfigure[Tablero 3]{\label{fig:layout4}\includegraphics[scale=0.2]{Boards/Layout4White.png}}
  \end{center}
  \caption{Distintos tableros}
  \label{fig:layouts}
\end{figure}
	
	
\begin{table}[h]
\begin{center}
	\begin{tabular}{|p{2.3cm}|p{2cm}|p{2cm}|p{2cm}|p{2cm}|p{4cm}|}
	\hline
	 Algoritmo & Nodos expandidos & Nodos en la frontera & Nodos generados & Profundidad de la Soluci\'on & Tiempo de Procesamiento\\
	\hline \hline
		 \multicolumn{6}{|c|}{Layout 1} \\
	\hline
	\textit{DFS} & 6 & 0 & 6 & 4 & 32ms \\
	\textit{BFS} & 7 & 0 & 6 & 4 & 10ms \\
	\textit{Produndizaci\'on Iterativa} & 7 & 7 & 19 & 4 & 12ms \\
	\hline
		 \multicolumn{6}{|c|}{Layout 2} \\
	\hline
	\textit{DFS} & 11 & 57 & 10 & 10 & 74ms \\
	\textit{BFS} &5944 & 34491 & 5944 & ? & 5' \\
	\textit{Produndizaci\'on Iterativa} & 7550 & 33571 & 5696 & ? & 5' \\
	\hline
		 \multicolumn{6}{|c|}{Layout 3} \\
	\hline
	\textit{DFS} & 73 & 1438 & 72 & 72 & 2' 708ms \\
	\textit{BFS} & 641 & 15705 & 641 & ? & 5' \\
	\textit{Produndizaci\'on Iterativa} & 620 & 14964 & 587 & ? & 5' \\
	\hline
	\end{tabular}
\end{center}
\caption{Resultados de resolver el problema de los tableros \ref{fig:layout1}, \ref{fig:layout3} y \ref{fig:layout4} con distintos algoritmos no informados}
\label{tab:cost}
\end{table}

\end{document}